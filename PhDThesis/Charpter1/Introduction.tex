
\chpt{Introduction}

% Cosmic ray
The story of cosmic rays began in 1912 when a balloon experiment carried out by Victor Hess showed a significant rise in the air ionization rate with increasing altitude, confirming the existence of cosmic rays \cite{NobelCosmicRay}. The discovery of cosmic rays opened a new window to explore our universe supplementing the astronomical observations. Since then, enormous effort has been put in to identify the components of cosmic rays and to measure their flux in a wide energy scale range. Due to the very high levels of energy that cosmic rays can reach, they provide a unique opportunity to study high energy particles with energy above the TeV scale and even higher than the energy levels reached by modern particle colliders.  \par   

% Antiproton and antiproton in cosmic rays
Before any particle accelerator was built, the main source to study high energy particles was the cosmic rays. The positron was firstly found in 1932 by Carl David Anderson in cosmic rays \cite{PositronAndersonPaper}. That was the first antiparticle to be discovered. After that, particle accelerators made great progress and the antiproton was discovered in 1955 by Emilio Segrè and Owen Chamberlain at the Bevatron particle accelerator \cite{AntiprotonDiscoverPaper}. The presence of antiprotons in cosmic rays was firstly confirmed in 1979 by two balloon experiments \cite{CosmicAntiprotonBogomolov1979, CosmicAntiprotonGolden1979}, the effort to study cosmic antiprotons has never stopped. \par 

% Antiproton measurement and AMS-02
Due to the Earth’s atmosphere, cosmic antiprotons can only be measured either by balloon experiments or space spectrometers. In this thesis, data collected by the AMS-02 experiment is used to determine the antiproton to proton flux ratio in time-averaged analysis and time-dependent analysis. The AMS-02 detector was installed on the International Space Station (ISS) in May 2011 and has a permanent magnet to distinguish the sign of the particles’ rigidity. Up to May 2021, cosmic ray data of ten years has been collected, and is used for the present analysis. \par    

% Time-dependent antiproton
The antiproton to proton flux ratio has been studied with the AMS-02 experiment and other experiments. Above 60 GV the antiproton to proton flux ratio is observed to be stable, rather than going down as predicted by traditional secondary production. This behavior indicates that an additional contribution like dark matter is needed or the cosmic ray propagation model needs to be revised. In this thesis, the updated antiproton to proton flux ratio will be given with the latest ten years dataset. \par

When the protons and antiprotons enter the heliosphere, they encounter a turbulent solar wind with an embedded heliospheric magnetic field (HMF), this leads to a time-dependent change in the antiproton to proton flux ratio. To observe this time structure in the flux ratio, long-time and precise measurements of cosmic protons and antiprotons are needed. Before the AMS-02 experiment, there was no continuous measurement of cosmic antiprotons. The balloon experiments measured cosmic antiprotons in several flights. The Space spectrometer PAMELA published time-averaged antiproton results \cite{PamelaAntiproton350Paper}. Since AMS-02 has been monitoring cosmic rays continuously up to today, the time-dependent antiproton analysis is possible to be done with the data collected. In this thesis, the time-dependent antiproton to proton flux ratio is presented in six Bartels Rotations time resolution, this is the first time that the long-time structure of the antiproton to proton flux ratio is observed. \par

% Chapters contents
Chapter \ref{ChpaterCosmicRays} presents a general picture of cosmic rays. An overview of the AMS-02 experiment and its sub-detectors is given in Chapter \ref{ChapterAMS02}. Chapter \ref{ChapterAnalysis} describes the antiproton analysis techniques used in this thesis. The result of time-averaged antiproton to proton flux ratio using ten years of AMS-02 data, as well as the time-dependent antiproton to proton flux ratio with a time resolution of six Bartels Rotations are presented in Chapter \ref{ChapterResult}. Chapter \ref{ChapterSummary} gives a summary and a conclusion on this thesis.  


%% Fabian 
\begin{comment}
%% Cosmic rays  
Since the discovery of cosmic rays by Victor Hess in 1912 [1], these high energy particles from space have led to many important discoveries about our universe. Particle physics in particular has been dominated by cosmic rays before man-made particle accelerators were built, for example resulting in the discovery of the positron by Carl Anderson in 1936 [2], the first antiparticle to be discovered. Even after particle accelerators were build, cosmic ray physics remained a topic of great interest, not merely because of the enormous energies reached by some cosmic ray particles, which are way out of reach for any modern laboratory. Most importantly, cosmic rays can teach us about the fundamental processes occurring in our galaxy and beyond. An overview of these processes and how they can accelerate and influence cosmic rays will be given in Chapter 2.
%% space experiment: AMS-02
Studying cosmic rays from the ground is challenging because the high energy particles interact in the atmosphere and only the fragments of these collisions can be detected on the ground. While this remains the only way to study cosmic rays at the highest energies due to the extremely low flux, the energy range from MeVs to TeVs is best explored by experiments above most of the atmosphere, either in balloons or using space-based experiments. The first test of a magnetic spectrometer in space was conducted in June 1998 on board of the space shuttle Discovery, the AMS-01 experiment [3]. Its successor AMS-02 was installed on the International Space Station in May 2011 [4] and is currently the only magnetic spectrometer measuring cosmic rays in space. Chapter 3 will give an overview of the AMS-02 experiment and all of its sub-detectors.
%% electron & positron: time-averaged and time-dependent
The electron and positron cosmic ray fluxes have been studied extensively with AMS-02 and many other experiments. As the lightest charged leptons, these particles are stable and can therefore reach us directly from their sources. An excess in the positron flux above ∼ 10 GeV [5], which cannot be explained by production of positrons in cosmic ray collisions, indicates that there is an additional source producing electrons and positrons. Low energy electrons and positrons are also of great interest because of the strong variation of their fluxes with time caused by changes in the heliosphere. Both short- and long-term variations can be observed in the fluxes, which have been previously published with 27 day time resolution [6] by AMS-02. In this thesis, the electron fluxes will be derived with daily time resolution, allowing a much more detailed understanding of the short term phenomena influencing cosmic rays in our Solar System.
% ECAL Baesd -- TRD Based
The published analysis of electrons and positrons with AMS-02 uses multiple sub- detectors including the electromagnetic calorimeter (ECAL), which is relatively small compared to the rest of the detector due to its large weight. This restricts the field of view available to such an analysis and therefore the available statistics. For the daily fluxes, it is essential to collect as much statistics as possible, even if this comes at the cost of larger systematic uncertainties, as long as these systematics are time-independent. Figure 1.1 illustrates the difference in geometric acceptance between an ECAL based analysis and a large acceptance one, which is based on the TRD and inner tracker acceptance. The geometric acceptance is about 6.5 times as large as the ECAL acceptance. Chapter 4 will describe this large acceptance analysis in detail and also compare the results of this thesis to the published ECAL based flux as well as the results from other working groups in the AMS collaboration.
% Result
Chapter 5 will show the preliminary fluxes from this analysis with 27 day and with daily time resolution. A detailed analysis of 19 Forbush decrease events based on the daily electron fluxes is shown as well.
% Conclusion
Chapter 6 will then give a conclusion of this thesis.
\end{comment}











