
\section{Sources of cosmic rays}
% Source of cosmic rays (explain why SNR can produce (see Fabian), and why it's power law) (Mertsch PPT)

%% Introduction
Traditionally, the supernova remnants (SNRs) are considered to be the main source of cosmic rays. This idea was firstly suggested by Baade and Zwicky in 1934 \cite{CosmicRaysFromSuperNova}. Apart from SNR, some other sources can also contribute to cosmic rays production, like pulsars can produce electron and positron pairs then accelerate them in the pulsar wind nebula (PWN), or dark matter can produce additional positrons and antiprotons from annihilation. Therefore, a high precision measurement of the cosmic ray flux is certainly a probe to understand the sources of cosmic rays. \par

%% supernova and supernova remnant 
A supernova is a stellar explosion event that can expel several solar masses of material with very high speed. There are mainly two mechanisms of supernova creation \cite{SupernovaClassification}: a) a white dwarf accretes matter from a companion star until it reaches the Chandrasekhar mass limit and then a nuclear fusion is triggered, and b) a massive star undergoes gravitational core collapse. During a supernova explosion, matter is ejected and accelerated at velocities of a few percent of the speed of light. As the ejected material travels faster than the speed of sound in the Interstellar Medium (ISM), it creates an expanding shock wave that sweeps up the interstellar material of gas and dust creating a "supernova remnant" \cite{SupernovaRemnants}. \par

%% Cosmic rays accelerated from the supernova remnant 
Charged particles can be absorbed into the supernova remnant and be confined in it. They gain energy and are accelerated while trapped in the supernova remnant until their energy is large enough to escape from it. The acceleration mechanisms are called "First Order Fermi Mechanism" and "Second Order Fermi Mechanism". The first one describes the particles crossing the shock front repeatedly gaining energy \cite{FirstFermiAcceleration}. The second one describes the acceleration from the particle collisions with magnetic clouds in the interstellar medium \cite{SecondFermiAcceleration}. These acceleration processes in the supernova remnant eventually lead to a power law energy spectrum.  \par

%% Pulsars generate supernova remnant  
Apart from the SNRs, pulsars can also produce high energy cosmic rays. Pulsars are rapidly rotating neutron stars with a strong magnetic field. A PWN is usually formed around a pulsar within the shell of a supernova remnant, where electrons and positrons can be produced by pair production \cite{PulsarGenerateCosmicRays, PulsarGenerateCosmicRays2}. The shock of the PWN with the surrounding matter can accelerate these charged particles to very high energies \cite{PulsarPWN, PulsarAccelerateCosmicRays}. This effect results in an important contribution to the cosmic electron and positron spectra in the high energy range. \par

%% Source of antiproton (Interactions and Dark Matter)
Since particles are produced via pair production and proton (antiproton) is much heavier than electron (positron), pulsars can produce positrons while antiprotons can not be produced by pulsars \cite{PulsarProducePositronOnly, PulsarProducePositronOnly2}. Therefore, the contribution of cosmic antiprotons to the cosmic ray spectrum, can only result from the interactions between the primary cosmic rays and the ISM. Based on this, some models give the predictions of the antiproton flux or antiproton to proton flux ratio purely from secondary production \cite{TimeAveragedPbarRatioPaperSecondaryProduction1, TimeAveragedPbarRatioPaperSecondaryProduction2}. By comparing the measurement of the cosmic antiproton, if an excess in the cosmic antiproton is observed, different sources of antiproton production, like dark matter should be investigated \cite{CosmicAntiprotons, CosmicAntiprotons2,CosmicAntiprotons3}. For example, dark matter can produce additional positrons and antiprotons via annihilation besides secondary production, the additional components from dark matter plus the secondary production could be used to fit the measured cosmic antiproton spectrum \cite{AntiprotonFromSecondaryAndDarkMatterModel}. In summary, by measuring the antiproton spectrum precisely, the sources of cosmic antiprotons can be studied.  
     
     
     
     
     
     