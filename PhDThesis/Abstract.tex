 
\vspace*{1.5cm}
\begin{center}
 \begin{huge}
    \textit{Abstract}
 \end{huge}
\end{center}
\begin{center}
\textbf{\normalsize{\mytitlename}}
\end{center}

%% Comic Antiprotons
The measurement of the cosmic antiproton to proton flux ratio is a very sensitive probe of the sources of cosmic rays and their propagation. Cosmic antiprotons are assumed to be mainly secondary cosmic rays from interactions of primary cosmic rays like protons with the interstellar medium.

In this thesis, the antiproton to proton flux ratio has been measured with the AMS-02 experiment on the ISS in the rigidity range from 1 to 525 GV with data recorded between 2011 and 2021. In the rigidity range below 10 GV, cosmic rays are significantly affected by the time variation of the solar magnetic field. Antiprotons and protons propagate differently in the solar magnetic field. Therefore, the long-term precision measurements by AMS-02 allow for a deeper understanding of charge sign dependent solar modulation effects.\par

For the first time, in the rigidity range up to 18 GV, the antiproton to proton flux ratio is shown with a time resolution of 162 days, corresponding to six Bartels Rotations. Unexpectedly, a different time structure is observed than in the electron to positron ratio below 3 GV. The significance of the time structure decreases with increasing rigidity, and no significant time structure is observed for rigidities above 10 GV.  \par

Antiprotons, like positrons, are sensitive probes for dark matter annihilation processes. No significant structures are observed in the time-averaged antiproton to proton flux ratio. The antiproton to proton flux ratio is expected to decrease for rigidities above $\sim$40 GV. The slope in this rigidity range was measured to be $(-1.44\pm 0.42)\times 10^{-5}$. 


%Unexpectedly, this could not be established with the AMS-02 data. 



