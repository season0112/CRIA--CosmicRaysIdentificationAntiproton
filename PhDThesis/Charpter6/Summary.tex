
%\section{Summary}


%% time-averaged result
In this thesis, the time-averaged antiproton to proton flux ratio as a function of rigidity is presented for up to 525 GV using ten years of AMS-02 cosmic ray data providing the most accurate and up to date measurement of this kind. This unprecedented accuracy provides a probe for us to understand the origin of cosmic ray antiprotons. The antiproton to proton flux ratio in the high rigidity range shows a relatively flat trend according to previous AMS-02 publication based on four years of data \cite{AMS02AntiprotonPRL2016}, instead of a falling trend predicted by the secondary production models \cite{TimeAveragedPbarRatioPaperSecondaryProduction1, TimeAveragedPbarRatioPaperSecondaryProduction2}. This observation doesn't change in this analysis result with the latest ten years of data. This analysis is an independent result of the AMS-02 Physics Report result in \cite{PhysicsReport2}, which also presents the antiproton to proton flux ratio up to 525 GV. 

%% time-dependent result
Apart from the time-averaged antiproton to proton flux ratio, the time-dependent antiproton to proton flux ratio in every six Bartels Rotations time bins is presented in this thesis. From 1.51 GV to 18 GV, the time-dependent antiproton to proton flux ratios are given in 13 rigidity bins. The rigidity bins are merged in every two rigidity bins of the time-averaged analysis to increase the statistics. \par

The first observation in the time-dependent antiproton to proton flux ratio is that with increasing rigidity, the solar modulation effect becomes weaker. The antiproton to proton flux ratio shows a distinct time structure for up to around 10 GV. The data was collected from May 2011 to May 2021, which covers eight years in the solar cycle 24 and two years in the solar cycle 25. During this ten-year period, the behavior of the antiproton to proton flux ratio shows a rising trend first, then goes down reaching a minimum, and finally it gradually rises. The importance of these observations is crucial as they can be used for the evaluation of the solar modulation models \cite{TimeDependentPbarRatioModelPaper, AslamModulationPaper}. 

The second observation in the time-dependent antiproton to proton flux ratio derives from the comparison between the antiproton to proton flux ratio and the electron to positron flux ratio. The antiproton to proton flux ratio is different from the electron to positron flux ratio in the rigidity range below 3 GV. This difference gradually fades away with increasing rigidity and above 6 GV it almost disappears and the two flux ratios show a similar trend. The different behaviors could be due to the different LIS of antiprotons and positrons in the low rigidity range. In addition, the mass difference between antiproton (proton) and positron (electron) also plays an important role in the low rigidity range.      
%therefore they are subject to different modulation effects. 

%% future
The AMS-02 will continue to collect data throughout the lifetime of the ISS, which is estimated to last until 2030. The increased statistics will reduce the statistical error further and give a more precise measurement. With the data collected in the future, the time variation effects in a complete 11 years solar cycle can be obtained. Moreover, studies on the antiproton to proton flux ratio in finer than six Bartels Rotations time bins will be made possible providing important insight on possible fine time structures present. \par
% Also, solar cycle 25 is expected until around 2030. 

   
%% AMS100
Preparations for the successor of the AMS-02 experiment are in progress. The AMS-100 \cite{AMS100}, a next-generation magnetic spectrometer in space is planned to be placed on the Lagrange point L2 of the Earth-Sun system in around 2039. The AMS-100 can provide 100 $m^2 sr$ acceptance and allow a measurement of the antiproton to proton flux ratio up to 10 TV. For the time-dependent analysis, the AMS-100 can continue the study of the solar modulation on the cosmic antiprotons for decades. 



\begin{comment}
% Nico:
%  time-averaged result
The presented time-averaged and time-dependent fluxes by AMS-02 are the most accurate measurements of the cosmic-ray electron and positron flux to date. The unprecedented accuracy in the data challenges our understanding of the origin of cosmic-ray positrons. The positron flux - at high energy - shows strong evidence for a source component responsible for the high-energetic positrons. For the first time a cut-off in the positron flux was measured, with a confidence of $4a$. The origin of the cut-off in the positron flux could be an astrophysical source, such as a pulsar. On the other hand the sharp drop-off of the flux could be the manifestation of a kinematic edge, related to dark matter annihilation. The electron flux shows no hint of a cut-off: it can be described by the sum of two power laws.
% source:  anisotropy
A key handle to differentiate between the dark matter and pulsar hypothesis is the measurement of the anisotropy in the arrival directions. The current limits on the dipole anisotropy of a < 0.019 at the 95\% confidence level are not competitive to rule out the pulsar origin. A novel large-acceptance analysis is under development, which will measure the fluxes as function of rigidity, not utilizing the ECAL. This allows one to increase the acceptance by a factor = 4.
% future
AMS-02 will continue to measure until the end of the ISS lifetime. Improvements in the analysis techniques [Kounine2017a] will allow us to measure the positron flux beyond the cut-off energy Es = 745+168 GeV -283, up to = 2 TeV and to determine the cut-off with more than $5a$ confidence. The model independent search for a spectral index change is currently limited by statistics. With the current dataset the break energy was determined to be E0 = 333+61 GeV and the change of the spectral -15 index to dy = -0.57 + 0.18. With the extended dataset a significance of more than $5a$ for the change of the spectral index and the break energy is in reach. Furthermore the gain in statistics and the large-acceptance analysis will allow us to probe the dipole anisotropy on the sub-percent level, allowing to detect either a signature of anisotropy or set the most stringent limits.
% time-dependent result
As already formulated in Ref. [Aguilar2018], for the first time, the charge-sign dependent modulation during solar maximum has been investigated in detail by electrons and positrons alone, using the time-dependent fluxes. Prominent, distinct, and coincident structures in both the positron flux and the electron flux on a time scale of months were identified, that are not visible in the e+/e- flux ratio. Instead a long-term feature in the e+/e- flux ratio was revealed: a smooth transition from one value to another, after the polarity reversal of the solar magnetic field. The transition magnitude is decreasing as a function of energy, consistent with expectations from solar modulation models including drift effects. This novel dataset provides accurate input to the understanding of solar modulation. In the past, flux models were often constructed above = 20 GeV, where the influence of solar modulation plays a minor role. Using the time-dependent precision data presented in this thesis, sophisticated models can be developed, incorporating charge-sign dependent solar modulation, that allows one to describe the electron and positron flux over the whole energy range from 0.5 GeV to 1 TeV.
% AMS100
Final insights on the origin of cosmic-ray positrons will be given by AMS-100 [Schael2019], which offers 1000 times the acceptance of AMS-02, and allows for a measurement of the electron and positron flux up to 10 TeV. The science program could start around the year 2040, which would mark the begin of a new era in astroparticle physics.
\end{comment}


\begin{comment}
% Fabian:
% time-averaged result
The cosmic ray electron sample determined in this thesis is the largest electron dataset in the range from 1.07 GV to 107 GV with ∼ 87.5 million events. This large statistics allows an accurate measurement of the electron flux on a daily basis, which reveals previously unseen time structures in the fluxes at low rigidities.
% large acceptance method
The analysis presented in this thesis uses the full ∼ 0.48 m2sr acceptance of the experiment rather than the much smaller ∼ 0.07 m2sr ECAL acceptance. Analysing electrons without the ECAL with AMS-02 is challenging because it removes the excellent energy resolution and the selection redundancy.
% 1. rigidity resolution allow up to 200GV.  2. unfolding correction lead to differences 
The energy scale used here is based on the rigidity measurement of the tracker. The resolution is dominated by multiple scattering and bremsstrahlung at low rigidities and by the resolution of the spectrometer – in particular for the inner tracker – towards higher rigidities, which limits this analysis to |R| 􏰿 200GV. A Bayesian unfolding procedure has been extensively tested on MC and is applied to obtain the final energy scale for this analysis. This procedure further limits the rigidity range in which a flux can be determined to roughly 1 − 100 GV. The unfolding is the largest correction in this analysis, reaching up to ∼ 70\% in some rigidity bins. The comparison to results of an ECAL based analysis shows that a correction as a function of rigidity is required to reach agreement between the fluxes.
% bias in the TRD efficiency corrections.
The selection of electron events usually makes heavy use of the ECAL, both directly and indirectly. All selection efficiencies are determined from data, which is possible due to the redundancy in the various sub-detectors of AMS-02. Without the ECAL, this redundancy is missing as electrons are largely identified by the TRD alone, which could result in a bias in the TRD efficiency corrections.
% uncertainties
The results of this thesis have shown that the systematic uncertainties are significantly larger compared to a conventional analysis using the ECAL. The larger statistics are however improving the accuracy of the daily electron fluxes. A comparison to the published fluxes in Bartels rotations [6] shows good agreement for the time-dependent behaviour of the flux.
% time-dependent result 
The daily electron fluxes have been determined in 12 bins from 1.07 to 61.7 GV and show significant time structures up to ∼ 10GV. Figure 6.1 shows an overview of the daily electron fluxes in five rigidity bins. While the long term behaviour of the flux seems to follow the 11-year solar cycle, 19 Forbush decreases have been found and analysed in detail. These events are caused by individual solar events (flares and CMEs). Most events show either an exponential or a linear decrease of intensity as a function of log(|R|) and a constant recovery time period as a function of rigidity. However, some events have been found to deviate from this behaviour.
% impact: first daily result
The results of this thesis are the first dataset of CR electron fluxes with daily time resolution in this rigidity range. Together with the AMS-02 daily proton fluxes [124], these results will allow a better understanding of how cosmic rays are affected by the constant changes in the heliosphere. However, more work will be required to understand the necessary corrections as a function of rigidity and to get better agreement between the different working groups in the AMS collaboration.
% future
AMS-02 will continue to take data until the end of the ISS lifetime, at least until 2028. This will result in the first cosmic ray dataset collected in space over an entire solar cycle. It is of particular interest to see whether or not the electron and positron fluxes will actually follow an 11-year cycle, as previously indicated in figure 5.7. Furthermore, the frequency of solar events causing Forbush decreases in the cosmic ray fluxes has reduced towards the solar minimum in December 2019. The next solar maximum is expected to occur in 2025 [125] and it will be interesting to see when and how Forbush decreases will start to appear again.
% AMS100
A successor experiment to AMS-02 is already in planning. The AMS-100 experiment [126] will be placed at the Sun-Earth Lagrange point L2 and have a 100 m2sr acceptance. This allows measurement of the electron and positron fluxes with much higher statistics up to ∼ 10 TeV. The launch is currently estimated for the year 2039.
\end{comment}





















