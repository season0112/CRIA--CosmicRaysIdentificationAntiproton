
\section{Antiproton To Proton Flux Ratio} \label{SecAntiprotonToProtonRatio}

%% Time Averaged Antiproton to proton flux ratio
%In this section, the formula used for the calculation of the antiproton to proton flux ratio, which is the objective of this thesis, is given. This formula is based on equation \ref{FluxCalculationEquation} that is used for the calculation of cosmic ray fluxes, as well as the equation \ref{EffectiveAcceptanceRatioCanceledOut} that defines the ratio between the proton and antiproton effective acceptances. All the ingredients needed have been described already in the previous sections. Given the antiproton and proton flux definitions:


In this section, the antiproton to proton flux ratio is given with statistical uncertainty only. According to equation \ref{PbarOverProtonRatioEquation}, the calculated time-averaged antiproton to proton flux ratio is given with statistical uncertainty only in figure \ref{TimeAveragedRatioWithOverLapping_StaErrOnly}. Since the antiprotons are determined from different template fits in three different rigidity ranges, the results in these two overlapping ranges are also given. In the two overlapping ranges, the results are consistent with each other. In figure \ref{TimeAveragedRatioNotOverLapping_StaErrOnly}, the time-averaged antiproton to proton flux ratio without overlapping is given. The overlapping range is dealt with the best matching results. Below 4.43 GV the result in the low rigidity range is used. From 4.43 to 15.3 GV the result in the intermediate range is used. Above 15.3 GV the result in the high rigidity range is used. In figure \ref{TimeAveragedStatisticalError}, the relative statistical uncertainty of the antiproton to proton flux ratio is shown. The systematic uncertainty study will be shown in the next section. \par
 
\begin{figure}[H]
\centering
\includegraphics[width=1.00\textwidth, height=0.36\textheight]{Figures/chapter4/AntiprotonToProtonFluxRatioCalculation/TimeAveraged/{fullratio_StaErrOnly_ThisAnallysisOnly_pass78}.pdf}
\caption[The time-averaged antiproton to proton flux ratio in three rigidity ranges.]{The time-averaged antiproton to proton flux ratio in three rigidity ranges. In the two overlapping ranges, the results from different template fit methods are consistent with each other. The error bars are the statistical uncertainty only.}
\label{TimeAveragedRatioWithOverLapping_StaErrOnly}
\end{figure}

\begin{figure}[H]
\centering
\includegraphics[width=1.00\textwidth, height=0.36\textheight]{Figures/chapter4/AntiprotonToProtonFluxRatioCalculation/TimeAveraged/{fullratio_StaErrOnly_NotOverlapped_ThisAnallysisOnly_pass78}.pdf}
\caption[The time-averaged antiproton to proton flux ratio with statistical uncertainty only.]{The time-averaged antiproton to proton flux ratio with statistical uncertainty only.}
\label{TimeAveragedRatioNotOverLapping_StaErrOnly}
\end{figure}

\begin{figure}[H]
\centering
\includegraphics[width=1.00\textwidth]{Figures/chapter4/AntiprotonToProtonFluxRatioCalculation/TimeAveraged/{StaRelErr_pass78}.pdf}
\caption[The relative statistical uncertainty of the antiproton to proton flux ratio.]{The relative statistical uncertainty of the antiproton to proton flux ratio.}
\label{TimeAveragedStatisticalError}
\end{figure}


%% Time Dependent Antiproton to proton flux ratio
For the time-dependent antiproton to proton flux ratio, the calculation is based on equation \ref{TimeDependentPbarOverProtonRatioEquation}. As an example, the antiproton to proton flux ratio with statistical uncertainty only in the rigidity range of 1.92 to 2.4 GV is shown in figure \ref{TimeDependentRatio_StaErrOnly}. The corresponding statistical uncertainty is given in \ref{ExampleTimeDependenttStatisticalError}.

\begin{figure}[H]
\centering
\includegraphics[width=1.00\textwidth, height=0.4\textheight]{Figures/chapter4/AntiprotonToProtonFluxRatioCalculation/TimeDependent/{6Bartels_binmerge2_1.92_2.4_StaErrOnly}.pdf}
\caption[The antiproton to proton flux ratio in 1.92 to 2.4 GV in six Bartels Rotation.]{The antiproton to proton flux ratio in the rigidity range of 1.92 to 2.4 GV in six Bartels Rotation time bin.}
\label{TimeDependentRatio_StaErrOnly}
\end{figure}

\begin{figure}[H]
\centering
\includegraphics[width=1.00\textwidth, height=0.4\textheight]{Figures/chapter4/AntiprotonToProtonFluxRatioCalculation/TimeDependent/{StatisticalRelativeError_6Bartels_binmerge_2_1.92_2.4}.pdf}
\caption[The relative statistical uncertainty of the antiproton to proton flux ratio in 1.92 to 2.4 GV in six Bartels Rotation.]{The relative statistical uncertainty of the antiproton to proton flux ratio in the rigidity range of 1.92 to 2.4 GV in six Bartels Rotation time bin.}
\label{ExampleTimeDependenttStatisticalError}
\end{figure}






