
\section{Trigger Efficiency} \label{TriggerEfficiencySection}

After the measuring time, the next ingredient needed to calculate the particle flux is the trigger efficiency. \par

%% Definition of Trigger Efficiency
The trigger efficiency is the probability for an incoming particle inside the geometric acceptance to activate the trigger system. In the AMS-02 experiment, the trigger logic is based on the response of the TOF, the ACC, and the ECAL. There are three stages in the AMS-02 trigger architecture: "Fast trigger", "Level 1 trigger" and "Level 3 trigger". The three stages are processed in sequence, namely the next stage is activated after the previous stage is fulfilled. As there is enough bandwidth to transfer the data to the ground, the "Level 3 trigger" is not activated.  \par

%% Fast trigger
There are two types of the "Fast trigger" \cite{ACCAsTrigger} stage. The first one is using the TOF. If the FTC and FTZ trigger decisions \cite{BastianPhDPaper} are fulfilled, the first type of fast trigger is set. The second one is using the ECAL. This kind of trigger is generated by showers detected by the ECAL. The first and second fast triggers are for charged particles and photons or leptons respectively. \par
%comment: the FTC:  the FTC one is generated if the CP signal (At least one TOF counter with either side having signals higher than the high threshold) is get from at least three out of the four TOF layers, the FTZ: the coincidence within 640 ns of the BZ (At least one counter with either side exceeding the super high threshold.) signals from all four layers

%% Level 1 trigger
The "Layer 1 trigger" consists of seven types of trigger conditions: \par

\begin{itemize}
\item Single charged: Has High Threshold signals (HT) in all four TOF layers, also no ACC hits. 

\item Fast ions: Has Super High Threshold (SHT) signals in all four TOF layers, also less than five ACC hits. (From 26 Feb 2016, the second condition changed to less than eight ACC hits to improve statistics.)

\item Slow ions: Has SHT signals in all four TOF layers within 640 ns. 

\item Electrons: Has HT signals in all four TOF layers, also requires at least two ECAL superlayers signals in both XZ and YZ planes. 

\item Photons: Has an ECAL shower with a zenith angle of less than $20^{\circ}$ in both XZ and YZ planes. 


\item Unbiased TOF:  Has at least three out of four TOF layer HT signals. The events triggered by this is prescaled with a factor of $f_{\mathrm{TOF}}$=100, in order to reduce the trigger rate and save bandwidth. 

\item Unbiased ECAL: Has signals in at least two ECAL superlayers in the X-Z or Y-Z plane. The events triggered by this is prescaled with a factor of $f_{\mathrm{ECAL}}$=1000, in order to reduce the trigger rate and save bandwidth. 

\end{itemize}

Among the seven types of Layer 1 triggers, the first five are called $\textit{physics triggers}$. In this analysis, only the physics triggered events are used for counting the signal numbers. To construct the flux, the last two unbiased non-physics triggers are used to calculate the trigger efficiency.

%% Calculation for double counting 
The two unbiased triggers can both fire at the same time, therefore the double-counted events should be corrected as:

\begin{equation}
\frac{1}{f_{\mathrm{TOF+ECAL}}} = \frac{1}{f_{\mathrm{TOF}}} + \frac{1}{f_{\mathrm{ECAL}}} - \frac{1} { f_{\mathrm{TOF}} \cdot f_{\mathrm{ECAL}} } 
\end{equation}

where the $f_{\mathrm{TOF}}$ and $f_{\mathrm{ECAL}}$ are the prescaling factors for unbiased TOF and unbiased ECAL triggers mentioned before, the double-counting events are prescaled with a factor of $f_{\mathrm{TOF+ECAL}} \approx 90.99$.

Once the prescaling factor is determined, the trigger efficiency $\epsilon_\mathrm{Trigger}$ can be calculated by comparing the number of physics trigger events and the unbiased non-physics trigger events:

\begin{equation}  
{\epsilon_\mathrm{Trigger}(R)}=\frac{N_\mathrm{Phys}(R)}{N_\mathrm{Phys}(R)+f_\mathrm{TOF} \cdot N_\mathrm{TOF}(R)+f_\mathrm{ECAL} \cdot N_\mathrm{ECAL}(R) + f_\mathrm{TOF+ECAL} \cdot N_\mathrm{TOF+ECAL}(R)}
\end{equation}

where $f_{\mathrm{TOF}}$, $f_{\mathrm{ECAL}}$ and $f_{\mathrm{TOF+ECAL}}$ are the prescaling factors, $N_\mathrm{Phys}$ is the number of physics trigger events, $N_\mathrm{TOF}$ and $N_\mathrm{ECAL}$ are the numbers of unbiased TOF and ECAL trigger events, $N_\mathrm{TOF+ECAL}$ is the number of events triggered by both unbiased TOF and ECAL trigger.

%% The trigger is directly taken from ISS data. \par
In figure \ref{TriggerEfficiency}, the trigger efficiency of proton is shown as an example. For the antiproton, the trigger efficiency is assumed to be the same.

\begin{figure}[H]
\centering
\includegraphics[width=1.0\textwidth, height=0.3\textheight]{Figures/chapter4/Trigger/{TriggerEff_noprescaling_B1042_antipr.pl1.1800_7.6_all_Pattern0}.pdf}
\caption{The trigger efficiency for proton events taken from MC.}
\label{TriggerEfficiency}
\end{figure}

\begin{comment}
Low and intermediate:
用的是真实的TriggerEfficiency, 该TriggerEfficiency通过“TriggerEfficiency”程序来计算得到。
    //// Load TriggerEfficiency //FIX ME: Load proton TriggerEff
    TFile *f5 = new TFile( (lowpath + string("/TriggerEff_B1042_antipr.pl1.1800_7.6_all.root")).c_str() );
    TFile *f6 = new TFile( (lowpath + string("/TriggerEff_B1042_antipr.pl1.1800_7.6_all.root")).c_str() );
    TH1F *Trig_antiproton_noprescaling_all = (TH1F*)f5->Get("TriggerEff_noprescaling");
    TH1F *Trig_proton_noprescaling_all       = (TH1F*)f6->Get("TriggerEff_noprescaling");
    TH1D *Trig_antiproton_noprescaling      = new TH1D("", "", 20, subrange_intermediate.data());
    TH1D *Trig_proton_noprescaling            = new TH1D("", "", 20, subrange_intermediate.data());
    
High:
用的是空的histogarm
实际“TriggerEfficiency”可以从Auxiliary中计算各种:通过“TriggerEfficiency”程序。
        PhysicsTriggerHisto->Write("PhysicsTriggerHisto");
        effMC_Preselection->Write("effMC_Preselection");
        effData_Preselection->Write("effData_Preselection");
        effMC_QualityCuts->Write("effMC_QualityCuts");
        effData_QualityCuts->Write("effData_QualityCuts");
        TriggerEff.Write("TriggerEff");
        TriggerEff_noprescaling.Write("TriggerEff_noprescaling");
       
\end{comment}

%% 1. Acceptance calculations do not use physics trigger cuts.  2. Trigger Efficiency canceled out.
Although the analysis is based on the physics triggered events, the cuts and selections do not include the physics trigger cut in the determination of effective acceptance in Section \ref{EffectiveAcceptanceSection}. The trigger efficiency has to be multiplied separately in the denominator of the flux. In the antiproton to proton flux ratio, the trigger efficiencies for proton $\epsilon_{p}$ and antiproton events $\epsilon_{ \overline{p} }$ cancel out in the calculation. However, the trigger efficiency is still needed in the unfolding procedure as it will be described in the next section.













