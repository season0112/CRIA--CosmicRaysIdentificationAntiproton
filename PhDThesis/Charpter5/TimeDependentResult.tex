
\section{Time Dependent Result}

%% Time-dependent Pbar to Proton Ratio
\begin{figure}[htpb]
\centering
\setlength{\abovecaptionskip}{-10mm}
\vspace{-5mm}
\includegraphics[width=0.8\textwidth, height=0.95\textheight]{Figures/chapter5/timedependent/low/{PbarRatioInAllRigidityBins6Bartels}.pdf}
%\vspace{-3mm}
\caption[Time-dependent antiproton to proton flux ratios for five bins.]{Time-dependent antiproton to proton flux ratios for five out of the 13 rigidity bins. The uncertainty bars in this figure are total time-dependent uncertainties, including the statistical uncertainty and the time-dependent systematic uncertainty. Distinct time variation structures are visible in these ratios.}
\label{timedependentrratioinMoreRigidity}
\end{figure}

%\clearpage

Apart from the time-averaged antiproton to proton flux ratio, the time-dependent antiproton to proton flux ratio is also calculated with the same strategy. For the time-dependent analysis, the antiproton to proton flux ratio is determined in six Bartels Rotations time bins. In total, 13 time-dependent antiproton to proton flux ratios are obtained from 1.51 GV to 18 GV. The first three rigidity bins from 1.0 to 1.51 GV in the time-averaged analysis are excluded since they deviate from the previous publication. The time-dependent antiproton to proton flux ratios for five out of the 13 rigidity bins are given in figure \ref{timedependentrratioinMoreRigidity}. The uncertainty bars in this figure are total time-dependent uncertainties, including the statistical uncertainty and the time-dependent systematic uncertainty. In this figure, the variation caused by the solar modulation is clearly visible.  \par

One important observation is that with increasing rigidity, the solar modulation effect becomes weaker. In the first rigidity bin in figure \ref{timedependentrratioinMoreRigidity}, the flux ratio changes by a factor $\sim$2 between its minimum and maximum values, while at around 10 GV, the fluctuation only changes by around 20\%.  \par

%% Trend
% From electron-positron PRL paper: 10.2012 to 03.2014 not well-defined polarity
A distinct trend is observed in the antiproton to proton flux ratio from 2011 to 2021. Below 4 GV, the antiproton to proton flux ratio shows a rising trend first and then gradually goes down, at last, relatively stabilized for a few years. The trend of antiproton to proton flux ratio in a complete solar cycle is shown in the solar modulation model \cite{TimeDependentPbarRatioModelPaper}. A similar trend is predicted in the corresponding current sheet tile angle period. In figure \ref{CompareWithModelPrediction}, the comparison between this result and the model predicted from \cite{AslamModulationPaper} is shown in an example rigidity bin of 1.92 to 2.4 GV. The observed behaviors match the solar modulation model in long time trend. Apart from the general trends observed over long periods of time, there are also a few fine time up and down structures. These fine time structures need to be studied in more detail with finer time bins. To achieve this, the cuts and selections in the low rigidity range need to be studied and optimized further dedicatedly. \par

%The solar cycle 24 began in Dec 2008 and ended in Dec 2019. The solar activity minimum was at early 2010. 
% Nico Ph.D. Thesis:
%In October 2011 and March 2012, there are sharp drops in the fluxes, followed by a quick recovery. The March 2012 event coincides with a strong Forbush decrease registered on March 8, 2012 [42]. Another drop occurred in August 2012; this was followed by an extended recovery period.
%In the period around July 2013 is the time of the solar magnetic field reversal. From May 2013 to April 2015, the flux of electrons continues to decrease, but with %reduced slope, while the positron flux begins to increase. Then, from April 2015 until May 2017, both fluxes rise steeply. The difference of the rate of the increase is %related to the charge-sign dependent solar modulation [15,43].

\begin{figure}[htpb]
\centering
\includegraphics[width=0.9\textwidth]{Figures/chapter5/timedependent/model/{CompareWithPbarOverProAslamModel1.92_2.4}.pdf}
\caption[Time-dependent antiproton to proton flux ratio compared with model.]{Time-dependent antiproton to proton flux ratio in 1.92 to 2.4 GV compared to model prediction. The modulation model is taken from \cite{AslamModulationPaper}, but the prediction curve in this figure is given in a finer time resolution of one Bartel Rotation. Prediction data provided by O.P.M. Aslam. This work and model prediction show a similar trend, though detailed comparison needs finer time resolution in the measured time-dependent antiproton to proton flux ratio.}
\label{CompareWithModelPrediction}
\end{figure}


%% Time-dependent Pbar to Proton Ratio Error
\begin{figure}[htpb]
\centering
\includegraphics[width=0.90\textwidth]{Figures/chapter5/timedependent/low/{RelativeErrorvsRigidity_6Bartels}.pdf}
\caption[Averaged relative statistical and systematic uncertainties of the time-dependent antiproton to proton flux ratio.]{Averaged relative statistical and systematic uncertainties of the time-dependent antiproton to proton flux ratio. Below 3 GV, statistical uncertainty is dominant. Above 3 GV, the contributions of statistical and systematic uncertainty are at a similar level. }
\label{AveragedRelativeError}
\end{figure}

The averaged relative uncertainty breakdown in time-dependent antiproton to proton flux ratio is shown in figure \ref{AveragedRelativeError}. Below 3 GV, the statistical uncertainty is dominant due to the limited statistics in six Bartels Rotations. Above 3 GV, the contributions of statistical and systematic uncertainty are at a similar level. For illustration, the relative total error breakdown in the rigidity bin of 1.92 - 2.4 GV is shown in figure \ref{TimeDependentErrorBreakDown}. The time-dependent systematic uncertainty is relatively stable but with some jumps due to the change in detector configuration. The statistical uncertainty is fluctuating mostly due to the antiproton event counts. \par


%In \ref{StatisticalErrorPercentage}, the statistical error contributions in total error in 1.92- 2.4 GV is shown, and the statistical error shows a relatively rising trend from 60\% to almost 80\%.
%%the statistical error shows a relatively stable percentage at around 50\% until 2018, since in 2018 the data taking decreased due to the frequent TTCS off.  \par

%\begin{figure}[t]
%\centering
%\includegraphics[width=1.0\textwidth, height=0.44\textheight]{Figures/chapter5/timedependent/low/{StaErrorPercentage_6Bartels_binmerge_2_1.92_2.4}.pdf}
%\caption{Percentage of statistical uncertainty in total uncertainty of the antiproton to proton flux ratio in 1.92 - 2.4 GV.}
%\label{StatisticalErrorPercentage}
%\end{figure}

%% Time-dependent Pbar to Proton Ratio Check
%In figure \ref{AveragedAndDependentRatioCompare}, the time-averaged antiproton to proton flux ratio and the 23 time-dependent antiproton to proton flux ratios are shown. The time-dependent antiproton to proton flux ratios have some time-dependent variations but mainly fluctuate around the time-averaged antiproton to proton flux ratio. 

%\begin{figure}[p]
%\centering
%\includegraphics[width=1.0\textwidth, height=0.4\textheight]{Figures/chapter5/timedependent/low/{TimeDependentAndAveragedRatio_vs_R_6B_binmerge2}.pdf}
%\caption{time-averaged antiproton to proton flux ratio (red) and 23 time-dependent antiproton to proton flux ratios in six Bartels rotations (green). The time-dependent results are changed around the time-averaged result.}
%\label{AveragedAndDependentRatioCompare}
%\end{figure}

%To compare the time-averaged and time-dependent antiproton to proton flux ratio, the 23 time-dependent antiproton and proton numbers in each rigidity bin are averaged respectively. Then the averaged over time antiproton to proton flux ratio is constructed. \par

%In figure \ref{MergedDependentResultAndAveraged}, the time-averaged antiproton to proton flux ratio and the averaged time-dependent antiproton to proton flux ratio are shown. From this comparison, the averaged time-dependent antiproton to proton flux ratio agrees well with the time-averaged antiproton to proton flux ratio, while in the first few rigidity bins, the averaged antiproton to proton flux ratio deviates due to the low statistics. \par 

%\begin{figure}[t]
%\centering
%\includegraphics[width=1.0\textwidth, height=0.4\textheight]{Figures/chapter5/timedependent/low/{MergedFitResult_vs_AveragedResult_6Bartels_binmerge2_residual}.pdf}
%\caption[Time-averaged and averaged time-dependent antiproton to proton flux ratio.]{Time-averaged antiproton to proton flux ratio (red) and averaged time-dependent antiproton to proton flux ratios in six Bartels rotations time bins (green). The averaged time-dependent result matches well with the time-averaged result.}
%\label{MergedDependentResultAndAveraged}
%\end{figure}


%% Compare with Electron/Positron
Since the solar modulation affects all kinds of cosmic rays, the obtained time-dependent antiproton to proton flux ratio can be compared with other results. In figure \ref{CompareWithElectron}, the antiproton to proton flux ratio in this analysis is shown together with the electron to positron flux ratio. The electron to positron flux ratio is taken from \cite{ZimmermannPhDThesis} and the result is consistent with a previous AMS-02 publication \cite{AMSElectronPositronPaper} but with a data extension. The electron to positron flux ratio is presented in its ECAL energy bins. To compare in the exact same rigidity bin as used for the antiproton to proton flux ratio, the electron and positron fluxes are fitted with a power-law modulated according to the force-field approximation, then integrated over the rigidity bin used for the antiproton to proton flux ratio and finally the electron to positron flux ratio in antiproton to proton flux ratio bins can be obtained. Due to the charge sign difference, the two flux ratios are shown in negatively charged particles over positively charged particles. The normalization of the electron to positron flux ratio is based on the first two years of data to make sure the mean of the electron to positron flux ratio is the same as the mean of the antiproton to proton flux ratio. \par
 
% Observation 1: With rigidity going up, the time variation effect decrease.
By comparing the antiproton to proton flux ratio and the electron to positron flux ratio, the first observation is that the time variation effects in both flux ratios are decreasing with increasing rigidity and this observation is expected.  \par

\begin{figure}[t]
\centering
\includegraphics[width=1.0\textwidth]{Figures/chapter5/timedependent/low/{RelativeBreakDownOfTotalError_6Bartels_binmerge_2_1.92_2.4}.pdf}
\caption[Breakdown of the total relative uncertainty in 1.92 to 2.4 GV.]{Breakdown of the total relative uncertainty of the antiproton to proton flux ratio into its statistical and systematic parts as a function of time for the rigidity bin of 1.92 to 2.4 GV. The statistical uncertainty corresponds to the antiproton events, which are subject to the solar modulation.}
\label{TimeDependentErrorBreakDown}
\end{figure}

% Observation 2: Pbar/Proton is different from Electron/Positron. And reasons: 1. Local interstellar spectrum. 2. Mass
% 1146: I don't recognize that the e-/e+ below 3 GV goes up with time, only the pbar/p ratio shows this trend.

The second observation is that the time variation trend of the antiproton to proton flux ratio is different from the electron to positron flux ratio in the low rigidity range. For example, below 3 GV, the antiproton to proton flux ratio goes up first and then drops down to the minimum, at last gradually rebounds. While the electron to positron flux ratio keeps relatively stable first and then goes down, at last rebounds from 2017. The difference between these two flux ratios gradually disappears as the rigidity increases, and the trend of these two ratios becomes similar in rigidity ranges higher than 6 GV.  \par

For the solar modulation modeling, the Local Interstellar Spectra (LIS) are important inputs. The LIS for the antiproton are different than the ones for the positron \cite{AslamModulationPaper}. Therefore, this could lead to a difference between the antiproton to proton flux ratio trend and the electron to positron flux ratio trend. Another important difference is the mass difference. The antiproton and the proton have much larger mass than the electron and the positron. Due to the drift effect, this mass difference has a higher impact in the low rigidity range resulting in differences between the flux ratio of the antiproton to proton and the one of the electron to positron.  \par 



 \begin{figure}[htpb]
    \subfigure[]{
        \includegraphics[width=1.0\textwidth, height=0.30\textheight ]{Figures/chapter5/timedependent/low/{Antiproton_Electron_Aachen_1.92_2.4_6Bartels}.pdf} 
    } 
    \subfigure[]{
        \includegraphics[width=1.0\textwidth, height=0.30\textheight ]{Figures/chapter5/timedependent/low/{Antiproton_Electron_Aachen_2.4_2.97_6Bartels}.pdf} 
    } 
\end{figure}

\begin{figure}[htpb]
    \subfigure[]{
	\includegraphics[width=1.0\textwidth, height=0.30\textheight ]{Figures/chapter5/timedependent/intermediate/{Antiproton_Electron_Aachen_4.43_5.37_6Bartels}.pdf}
    }
    \subfigure[]{
	\includegraphics[width=1.0\textwidth, height=0.30\textheight ]{Figures/chapter5/timedependent/intermediate/{Antiproton_Electron_Aachen_6.47_7.76_6Bartels}.pdf}
    }    
    \caption[Comparison between the antiproton to proton flux ratio and the electron to positron flux ratio over time in four rigidity bins.]{Comparison between the antiproton to proton flux ratio (blue) and the electron to positron flux ratio (red) over time in four rigidity bins: a) 1.92-2.4 GV b) 2.4-2.97 GV c) 4.43-5.37 GV and d) 6.47-7.76 GV. The error bars are the total errors calculated from the quadratic sum of the statistical errors and the total systematic errors. The scaling of the electron to positron flux ratio is based on the first two years of data to make sure the mean of the electron to positron flux ratio is the same as the mean of the antiproton to proton flux ratio.}
    \label{CompareWithElectron}
\end{figure} 







